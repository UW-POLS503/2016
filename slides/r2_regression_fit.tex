\documentclass[ignorenonframetext,]{beamer}
\setbeamertemplate{caption}[numbered]
\setbeamertemplate{caption label separator}{: }
\setbeamercolor{caption name}{fg=normal text.fg}
\usepackage{lmodern}
\usepackage{amssymb,amsmath}
\usepackage{ifxetex,ifluatex}
\usepackage{fixltx2e} % provides \textsubscript
\ifnum 0\ifxetex 1\fi\ifluatex 1\fi=0 % if pdftex
  \usepackage[T1]{fontenc}
  \usepackage[utf8]{inputenc}
\else % if luatex or xelatex
  \ifxetex
    \usepackage{mathspec}
  \else
    \usepackage{fontspec}
  \fi
  \defaultfontfeatures{Ligatures=TeX,Scale=MatchLowercase}
  \newcommand{\euro}{€}
\fi
% use upquote if available, for straight quotes in verbatim environments
\IfFileExists{upquote.sty}{\usepackage{upquote}}{}
% use microtype if available
\IfFileExists{microtype.sty}{%
\usepackage{microtype}
\UseMicrotypeSet[protrusion]{basicmath} % disable protrusion for tt fonts
}{}
\newif\ifbibliography
\usepackage{graphicx,grffile}
\makeatletter
\def\maxwidth{\ifdim\Gin@nat@width>\linewidth\linewidth\else\Gin@nat@width\fi}
\def\maxheight{\ifdim\Gin@nat@height>\textheight0.8\textheight\else\Gin@nat@height\fi}
\makeatother
% Scale images if necessary, so that they will not overflow the page
% margins by default, and it is still possible to overwrite the defaults
% using explicit options in \includegraphics[width, height, ...]{}
\setkeys{Gin}{width=\maxwidth,height=\maxheight,keepaspectratio}

% Prevent slide breaks in the middle of a paragraph:
\widowpenalties 1 10000
\raggedbottom

% Comment these out if you don't want a slide with just the
% part/section/subsection/subsubsection title:
\AtBeginPart{
  \let\insertpartnumber\relax
  \let\partname\relax
  \frame{\partpage}
}
\AtBeginSection{
  \ifbibliography
  \else
    \let\insertsectionnumber\relax
    \let\sectionname\relax
    \frame{\sectionpage}
  \fi
}
\AtBeginSubsection{
  \let\insertsubsectionnumber\relax
  \let\subsectionname\relax
  \frame{\subsectionpage}
}

\setlength{\emergencystretch}{3em}  % prevent overfull lines
\providecommand{\tightlist}{%
  \setlength{\itemsep}{0pt}\setlength{\parskip}{0pt}}
\setcounter{secnumdepth}{0}

\title{R-squared}
\author{Jeffrey Arnold}
\date{May 10, 2016}
\usepackage{amsmath}
\usepackage{amsfonts}
\DeclareMathOperator{\E}{E}
\DeclareMathOperator{\mean}{mean}
\DeclareMathOperator{\Var}{Var}
\DeclareMathOperator{\Cov}{Cov}
\DeclareMathOperator{\Cor}{Cor}
\DeclareMathOperator{\Bias}{Bias}
\DeclareMathOperator{\MSE}{MSE}
\DeclareMathOperator{\sd}{sd}
\DeclareMathOperator{\se}{se}
\DeclareMathOperator{\rank}{rank}
\DeclareMathOperator*{\argmin}{arg\,min}
\DeclareMathOperator*{\argmax}{arg\,max}

\newcommand{\mat}[1]{\boldsymbol{#1}}
\renewcommand{\vec}[1]{\boldsymbol{#1}}
%\renewcommand{\T}{'}

\newcommand{\distr}[1]{\mathcal{#1}}
\newcommand{\dnorm}{\distr{N}}
\newcommand{\dmvnorm}[1]{\distr{N}_{#1}}

\begin{document}
\frame{\titlepage}

\begin{frame}{R-squared, similar model fit stats, and advice on what to
do}

\begin{enumerate}
\def\labelenumi{\arabic{enumi}.}
\tightlist
\item
  R-squared
\item
  Adjusted R-squared
\item
  Standard error of the regression
\item
  F-test
\item
  Advice
\end{enumerate}

\end{frame}

\begin{frame}{Several definitions of \(R^2\)}

\begin{itemize}
\item
  Ratio of variance of fitted values to sample \(y\) \[
  R^2 = \frac{\Var(\hat{\vec{y}})}{\Var{\vec{y}}}
  \]
\item
  Ratio of variance ``explained'' by the regression \[
  R^2 = 1 - SSE / SST = 1 - \frac{\sum (y_i - \hat{y}_i)^2}{\sum (y_i - \bar{\vec{y}})^2}
  \]
\item
  For bivariate regression, correlation of \(Y\) and \(X\) squared, \[
  R^2 = \Cor(\vec{x}, \vec{y})^2 = \hat{\beta}_1 \frac{\sd{\vec{y}}}{\sd{\vec{x}}}
  \]
\item
  \(R^2 \in [0, 1]\) where \(1\) is all points are on a line/plane
\end{itemize}

\end{frame}

\begin{frame}{R-squared is dependent on scale of \(X\)}

\includegraphics{r2_regression_fit_files/figure-beamer/unnamed-chunk-2-1.pdf}

\(\hat{\sigma}^2 = 0.3\), \(R^2 = 0.91\)

\end{frame}

\begin{frame}{R-squared is dependent on scale of \(X\)}

Same data, regression on subset

\includegraphics{r2_regression_fit_files/figure-beamer/unnamed-chunk-3-1.pdf}

\(\hat{\sigma}^2 = 0.29\), \(R^2 = 0.75\)

\end{frame}

\begin{frame}{In-sample fit always increases as variables are added}

\(y = x + \epsilon\), \(\epsilon_i \sim N(0, 2)\)

\includegraphics{r2_regression_fit_files/figure-beamer/unnamed-chunk-5-1.pdf}

\end{frame}

\begin{frame}{\(R^2\) always increases as variables are added}

\includegraphics{r2_regression_fit_files/figure-beamer/unnamed-chunk-6-1.pdf}

\end{frame}

\begin{frame}{Other problems with R\^{}2}

\begin{enumerate}
\def\labelenumi{\arabic{enumi}.}
\tightlist
\item
  Does not measure goodness of fit

  \begin{enumerate}
  \def\labelenumii{\arabic{enumii}.}
  \tightlist
  \item
    To get \(R^2\) large, make \(X\) spread out
  \item
    To get \(R^2\) small, make \(X\) not spread out
  \end{enumerate}
\item
  Does not measure prediction
\item
  Cannot compare different datasets (including transformed \(Y\))
\item
  Not variance ``explained'' in causal sense
\end{enumerate}

\end{frame}

\begin{frame}{Adjusted \(R^2\)}

Adjust \(R^2\) for sample size and variables, \[
R^2 = 1 - \frac{SSE / (N - K - 1)}{SST / (N - 1)}
\]

\begin{itemize}
\tightlist
\item
  Slightly penalizes \(R^2\) for more variables
\item
  Adjustment only relevant for cases where \(N \approx K\)
\item
  Atheoretical
\item
  Doesn't fix any important problem with \(R^2\).
\item
  Pointless for comparing models
\end{itemize}

\end{frame}

\begin{frame}{Standard error of the regression (\(\hat{sigma}\))}

\[
\hat{\sigma} = \sqrt{ \frac{1}{N - K - 1} \sum \varepsilon_i^2 }
\]

\begin{itemize}
\tightlist
\item
  ``Average'' error
\item
  RMSE is similar, with denominator \(N\) instead of \(N - K - 1\).
\item
  On the same scale as \(\vec{y}\) - substantive interpretation
\item
  Often suggested as alternative to \(R^2\)
\end{itemize}

\end{frame}

\begin{frame}{Problems with \(\hat{\sigma}\)}

\begin{enumerate}
\def\labelenumi{\arabic{enumi}.}
\setcounter{enumi}{1}
\tightlist
\item
  All insample problems with \(R^2\) apply to \(\hat{\sigma}\)
\item
  To interpret \(\hat{\sigma}\) need to compare to scale (variance) of
  \(\vec{y}\), but then almost the same as \(R^2\).
\end{enumerate}

\end{frame}

\begin{frame}{\(F\)-test}

\begin{itemize}
\tightlist
\item
  \(R^2\) and \(\hat{\sigma}\) are statistics, but generally not used in
  tests
\item
  \(F\)-test with \(H_0: \beta_1 = \dots = \beta_K = 0\)
\item
  \(F\)-statistic is a function of the SSE of models
\item
  Inherits most of the same problems as \(R^2\)
\item
  Assumes that linear model is correct, not whether it is a good model
\end{itemize}

\end{frame}

\begin{frame}{What to do about it?}

\begin{enumerate}
\def\labelenumi{\arabic{enumi}.}
\tightlist
\item
  Focus on what's important:

  \begin{enumerate}
  \def\labelenumii{\arabic{enumii}.}
  \item
    If prediction: out of sample performance
  \item
    If causation:

    \begin{itemize}
    \tightlist
    \item
      identification of \(\beta\) (omitted variable bias or design)
    \item
      assumptions of model (other diagnostics)
    \end{itemize}
  \end{enumerate}
\item
  Focus on results/average of many models - not the ``best'' model
\end{enumerate}

\end{frame}

\begin{frame}{Next time}

Comparing predictive performance of models using cross-validation

\end{frame}

\begin{frame}{References}

\begin{itemize}
\tightlist
\item
  Gary King ``\href{http://gking.harvard.edu/files/mist.pdf}{How Not to
  Lie With Statistics: Avoiding Common Mistakes in Quantitative
  Political Science}.''
\item
  Cosmo Shalizi,
  \href{http://www.stat.cmu.edu/~cshalizi/mreg/15/lectures/10/lecture-10.pdf}{F-Tests,
  R2, and Other Distractions}.
\item
  Gelman and King.
  \href{http://andrewgelman.com/2007/08/29/rsquared_useful/}{R-squared:
  useful or evil?}
\end{itemize}

\end{frame}

\end{document}
